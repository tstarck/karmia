\documentclass[11pt]{article}
\usepackage[pdftex]{graphicx}
\usepackage[finnish]{babel}
\usepackage[utf8]{inputenc}
\usepackage{amsfonts}
\usepackage{amsmath}
\usepackage{amssymb}
\usepackage{color}

\begin{document}

\title{\Huge{\bf Toteutusdokumentti} \\ \Large{Karmia käärmetietokantasovellus}}
\author{Tuomas Starck}
\maketitle

\vspace{4em}

\section{Johdanto}

\subsection{Järjestelmän tarkoitus}

\paragraph{} Järjestelmä on käärmetietokanta, joka toimii apuvälineenä käärmeiden välityksessä. Tietokantaan sisältyy tiedot saatavilla olevista käärmeyksilöistä ja -lajeista ominaisuuksineen sekä käyttäjän mahdollisuus lainata itselleen käärme ja palauttaa se.

\subsection{Toimintaympäristö}

\paragraph{} Ohjelmisto jakaantuu sekä palvelimen että käyttäjän puolella ajettaviin komponentteihin, jotka tukevat toisiaan. Eri komponentit keskustelevat HTTP-käytännön välityksellä, joten palvelimen tulee olla yhteensopiva HTTP-palvelin ja käyttäjän ohjelmaympäristö on selain.

\section{Ohjelmiston yleisrakenne}

\section{Järjestelmän komponentit}

\section{Asennustiedot}

\begin{enumerate}
\list Karmian asennus ja käyttöönotto onnistuu helpoiten lataamalla ohjelmisto Githubista. \\ $ git clone git://url/ $
\list foo
\list bar
\end{enumerate}

\begin{verbatim}
karmia
├── auth.php
├── common.php
├── config
├── data.sql
├── doc
│   ├── kaliflow.jpg
│   ├── karmia.txt
│   ├── suunnittelu.tex
│   ├── tietokantakuva.jpg
│   ├── toteutus.tex
│   └── tree.txt
├── icon.png
├── index.php
├── isohali.css
├── isohali.js
├── isohali.php
├── json.css
├── json.js
├── json.php
├── json.xhtml
├── kaarme.php
├── karmia.sql
├── kirjaudu.php
├── kirjaudu.xhtml
├── linkit.js
├── linkit.php
├── lomake.css
├── lomake.js
├── main.css
├── oma.php
├── pgdb.php
├── pois.php
├── README
├── sql.php
├── TODO
├── uusi
│   ├── kayttaja.php
│   └── kayttaja.xhtml
└── xhtml.php
\end{verbatim}

\section{Käynnistys- ja käyttöohje}

% \subsection{}

% \begin{description}
% \item[] \hfill \\
% \end{description}

% \begin{figure}
% \caption{}
% \includegraphics[]{kuva.png}
% \end{figure}

\end{document}
