\documentclass[11pt]{article}
\usepackage[finnish]{babel}
\usepackage[utf8]{inputenc}
\usepackage{amsfonts}
\usepackage{amsmath}
\usepackage{amssymb}
\usepackage{color}

\begin{document}

\title{\Huge{\bf Karmia} \\ \large{Käärmetietokantasovellus}}
\author{Tuomas Starck ja Hanna Vuorivirta}
\maketitle

\vspace{4em}

\section{Johdanto}

\subsection{Järjeselmän tarkoitus}

\paragraph{} Työn aihe on käärmetietokanta, joka toimii apuvälineenä käärmeiden välityksessä. Tietokantaan sisältyy tiedot saatavilla olevista käärmeyksilöistä ja -lajeista ominaisuuksineen sekä käyttäjän mahdollisuus lainata itselleen käärme ja palauttaa se.

\subsection{Toimintaympäristö}

\paragraph{} Ohjelmisto jakaantuu sekä palvelimen että käyttäjän puolella ajettaviin komponentteihin, jotka tukevat toisiaan. Eri komponentit keskustelevat HTTP-käytännön välityksellä, joten palvelimen tulee olla yhteensopiva HTTP-palvelin ja käyttäjän ohjelmaympäristö on selain.

\subsection{Rajaukset}

\paragraph{} Järjestelmä käyttää yksinkertaistettua mallia salasanojen käsittelyyn, joka ei ole riittävän turvallinen suojaamaan käyttäjän salasanatietoja sellaisessa tapauksessa, jossa hyökkääjä pääsee käsiksi palvelinpuolen tietoihin. Järjestelmää ei siis suositella yleiseen käyttöön ilman muutoksia.

\paragraph{} Järjestelmä ei ole erityisesti suunniteltu skaalautumaan suuriin tieto- tai käyttäjämääriin. Esimerkiksi kaikkien maapallon käärmeiden lukumäärä voi olla liian suuri, jotta kaikkien käärmeiden tiedot olisivat järjestelmän hallittavissa.

\subsection{Toteutusympäristö}

\paragraph{} En tiedä, mitä sieniä pitää syödä, jotta onnistuu kirjoittamaan ohjeisiin sanan \emph{toteutusympäristö} selitykseksi: ''Missä ympäristössä työ toteutetaan.'' Minustakin olisi hauska kirjoittaa sanakirja tähän tyyliin:

\begin{description}
\item[femme fatale] Sellainen femme, joka on fatale.
\item[fennisti] Se, mikä fennisti on.
\item[fenomenaalinen] Se, mitä voi kuvailla fenomenaaliseksi.
\end{description}

\paragraph{} Ja niin edelleen. Mutten tiedä, mitä Laine tarkoittaa ympäristöllä, jossa työ toteutetaan, joten\ldots

\section{Yleiskuva järjestelmästä}

\subsection{Sidosryhmäkaavio}

% jokin kuva kai tähän

\subsection{Käyttäjäryhmät}

\paragraph{Tunnistautumaton} Tunnistautumaton on kuka tahansa, joka ei ole vielä kirjautunut järjestelmään tai luonut itselleen käyttäjätunnusta.
\paragraph{Käyttäjä} Käyttäjä on järjestelmän tunnistama henkilö.
\paragraph{Ylläpeto} Ylläpeto on käyttäjä, jolla on erityisiä oikeuksia hallita järjestelmän tietosisältöä.

\section{Käyttötapaukset}

\subsection{Tavalliset käyttötapaukset}

\begin{description}
\item[Uuden tunnuksen luonti] \hfill \\
Tunnistautumaton käyttäjä voi luoda itselleen käyttöluvan järjestelmään antamalla haluamansa tunnuksen ja salasanan. Jo aiemmin käytössä olevan tunnuksen antaminen ei johda onnistuneeseen uuden tunnuksen luontiin.
\item[Tunnistautuminen] \hfill \\
Tunnistautumaton käyttäjä kirjautuu järjestelmään.
\item[Käärmetietojen selaus] \hfill \\
Järjestelmä näyttää käyttäjälle järjestelmän tuntemat käärmeet niihin liittyvine tietoineen mukaanlukien mahdollisuuden lainata tai palauttaa käärmeen.
\item[Lainaa käärme] \hfill \\
Käyttäjä lainaa saatavilla olevan käärmeen ja järjestelmä merkitsee käärmeen käyttäjälle lainaan.
\item[Palauta käärme] \hfill \\
Käyttäjä palauttaa lainassa olevan käärmeen ja järjestelmä merkitsee käärmeen saataville.
\item[Tarkastele lainahistoriaa] \hfill \\
Järjestelmä näyttää käyttäjälle käyttäjän oman lainahistorian (tai osan siitä).
\end{description}

\subsection{Etuoikeutetut käyttötapaukset}

\begin{description}
\item[Lisää käärme] \hfill \\
Ylläpeto lisää käärmeen antamalla käärmeen nimen ja lajin. Järjestelmä luo uuden käärmeen, jonka tietoja voi selata ja jonka voi lainata.
\item[Lisää laji] \hfill \\
Ylläpeto lisää lajin antamalla lajin tarvittavat tiedot. Järjestelmä luo lajin, jonka tiedot voi liittää käärmeeseen.
\item[Käyttäjän oikeuksien ylennys] \hfill \\
Ylläpeto ylentää ei-ylläpeto-käyttäjän käyttöoikeudet ylläpeto-tasolle.
\item[Käyttäjän poistaminen] \hfill \\
Ylläpeto poistaa käyttäjän, jonka jälkeen kyseisellä käyttäjällä ei ole enää oikeutta käyttää järjestelmää.
\item[Käärmeen poistaminen] \hfill \\
Ylläpeto poistaa käärmeen, jolloin käärmeestä ei ole enää saatavilla tietoja eikä sitä voi lainata tai palauttaa.
\item[Lajin poistaminen] \hfill \\
Ylläpeto poistaa lajin, jolloin lajin tietoja ei voi enää liittää käärmeeseen.
\end{description}

\section{Tietosisältö}

\subsection{}

\paragraph{}

% . % . % . % . % . % . % . % . % . % . % . % . % . % . % . % . % . % . % . % . %
\section{}
\subsection{}
\paragraph{}
% . % . % . % . % . % . % . % . % . % . % . % . % . % . % . % . % . % . % . % . %

\end{document}
