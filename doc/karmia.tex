\documentclass[11pt]{article}
\usepackage[finnish]{babel}
\usepackage[utf8]{inputenc}
\usepackage{amsfonts}
\usepackage{amsmath}
\usepackage{amssymb}
\usepackage{color}

\begin{document}

\title{\Huge{\bf Karmia} \\ \large{Käärmetietokantasovellus}}
\author{Tuomas Starck ja Hanna Vuorivirta}
\maketitle

\vspace{4em}

\section{Johdanto} % 2p

\subsection{Aiheen kuvaus} % 1p

\paragraph{} Työn aihe on myrkkykäärmetietokanta, joka toimii apuvälineenä myrkkykäärmeiden välityksessä. Tietokantaan sisältyy tiedot saatavilla olevista myrkkykäärmelajeista ja yksilöistä tiettyine ominaisuuksineen sekä käyttäjän mahdollisuus varata itselleen käärme ja palauttaa se.

\subsection{Ympäristö} % 1p

\section{Suunnitteludokumentti} % 11p

\subsection{Määrittely} % 5p

\paragraph{Yleiskuva} Vain järjestelmänvalvoja pystyy hallinnoimaan tietokannan sisältöä eli lisäämään uuden käärmelajin, uuden yksilön ja näille ominaisuuksia sekä vastaavasti poistamaan ominaisuuksia, yksilön tai lajin. Käyttäjä pystyy selaamaan saatavilla olevia lajeja ja yksilöitä, etsimään niitä tietyn tai tiettyjen ominaisuuksien perusteella sekä luomaan itselleen tunnuksen, jolla voi varata mieleisensä käärmeen käyttöönsä.

\paragraph{Käyttötapaukset} % 3p

\begin{itemize}
\item tunnistautuminen
\item lisää käärme: järjestelmänvalvoja voi lisätä uuden käärmeen tietokantaan
\item lisää käärmeelle ominaisuus: järjestelmänvalvoja voi lisätä käärmeelle uuden ominaisuuden
\item poista käärmeeltä ominaisuus: järjestelmänvalvoja voi poistaa käärmeeltä ominaisuuden
\item poista käärme: järjestelmänvalvoja voi poistaa käärmeen tietokannasta
\item haku ominaisuuden perusteella
\item luo käyttäjä
\item poista käyttäjä
\item lainaa käärme
\item palauta käärme
\item käärmeyksilön lainahistorian tulostus
\item lainaajan lainahistorian tulostus
\end{itemize}

\subsection{Tietosisältö} % 4p

\begin{itemize}
\item taulu Käyttäjät: ( id, nimi, ylläpeto )
\item taulu Käärmeet: ( id, nimi, laji )
\item taulu Laji: ( id, väri, alkuperä, myrkyllisyys, uhanalaisuus, agressiivisuus )
\item taulu Lainat: ( id, käärme, lainaaja, lainan alku, lainan loppu )
\end{itemize}

\subsection{Käyttöliittymähahmotelma} % 2p

\paragraph{} Aluksi ohjelma pyytää käyttäjätunnusta ja salasanaa, jos sellaisia ei ole, ne voidaan luoda kirjautumisen yhteydessä. Kirjauduttuaan, käyttäjä voi listata tietokannassa olevat käärmeet klikkaamalla 'kaikki käärmeet' tai hakea käärmeitä tietyn tai tiettyjen ominaisuuksien perusteella valitsemalla ohjelman tulostamista ominaisuuksista. Esimerkiksi klikkaamalla auki 'väri'-kohdan, ohjelma tuottaa listan tietokannassa olevien käärmeiden väreistä, joista käyttäjä voi rastittaa mieleisensä lähempää tarkastelua varten. Hakua voi rajata useamman ominaisuuden perusteella yhtä aikaa. Yksittäisen käärmeen kohdalla voi 'lainaa'-painikkeella lainata käärmeen omalle käyttäjä\-tunnuk\-selleen ja vastaavasti merkitä käärmeen palautetuksi 'palauta'-painikkeella.

\section{Toteutusdokumentti} % 11p

\subsection{Ohjelmiston yleisrakenne} % 4p

\paragraph{} Ohjelmiston pääasiallinen logiikka toteutetaan PHP:lla. Muita käytössä olevia tekniikoita ovat SQL, XHTML, CSS ja Javascript sopivissa suhteissa.

\subsection{Järjestelmän komponentit} % 5p

\subsection{Asennustiedot} % 1p

\subsection{Käyttöohjeet} % 1p

\section{SQL} % 6p

\subsection{Create Table -lauseet} % 5p

\begin{verbatim}

CREATE TABLE kayttajat (
  tunnus varchar(8) PRIMARY KEY,
  salasana varchar(40) DEFAULT '',
  yllapeto boolean DEFAULT 'false'
);

CREATE TABLE kaarmeet (
  id int PRIMARY KEY,
  nimi varchar(60),
  laji int DEFAULT 0
);

CREATE TABLE lajit (
  id int PRIMARY KEY,
  nimi varchar(40),
  latin varchar(100),
  alkupera int,
  vari int,
  myrkyllisyys int,
  aggressiivisuus int,
  uhanalaisuus varchar(2)
);

CREATE TABLE lainat (
  id int PRIMARY KEY,
  kaarme varchar(40),
  lainaaja varchar(8),
  alku timestamp DEFAULT CURRENT_TIMESTAMP,
  loppu timestamp
);

CREATE TABLE alkupera (
  id int PRIMARY KEY,
  alkupera varchar(20)
);

CREATE TABLE vari (
  id int PRIMARY KEY,
  vari varchar(20)
);

CREATE TABLE myrkyllisyys (
  id int PRIMARY KEY,
  myrkyllisyys varchar(20)
);

CREATE TABLE aggressiivisuus (
  id int PRIMARY KEY,
  aggressiivisuus varchar(20)
);

\end{verbatim}

\subsection{SQL-hienosäätö} % 1p

\end{document}
